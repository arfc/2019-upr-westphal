%%%%%%%%%%%%%%%%%%%%%%%%%%%%%%%%%%%%%%%%%
% Jacobs Landscape Poster
% LaTeX Template
% Version 1.1 (14/06/14)
%
% Created by:
% Computational Physics and Biophysics Group, Jacobs University
% https://teamwork.jacobs-university.de:8443/confluence/display/CoPandBiG/LaTeX+Poster
% 
% Further modified by:
% Nathaniel Johnston (nathaniel@njohnston.ca)
%
% This template has been downloaded from:
% http://www.LaTeXTemplates.com
%
% License:
% CC BY-NC-SA 3.0 (http://creativecommons.org/licenses/by-nc-sa/3.0/)
%
%%%%%%%%%%%%%%%%%%%%%%%%%%%%%%%%%%%%%%%%%

%----------------------------------------------------------------------------------------
%	PACKAGES AND OTHER DOCUMENT CONFIGURATIONS
%----------------------------------------------------------------------------------------

\documentclass[final]{beamer}

\usepackage[scale=1.0]{beamerposter} % Use the beamerposter package for laying out the poster
\usetheme{confposter} % Use the confposter theme supplied with this template

\setbeamercolor{block title}{fg=dblue!80,bg=white} % Colors of the block titles
\setbeamercolor{block body}{fg=black,bg=white} % Colors of the body of blocks
\setbeamercolor{block alerted title}{fg=white,bg=dblue!70} % Colors of the highlighted block titles
\setbeamercolor{block alerted body}{fg=black,bg=dblue!10} % Colors of the body of highlighted blocks
% Many more colors are available for use in beamerthemeconfposter.sty

%-----------------------------------------------------------
% Define the column widths and overall poster size
% To set effective sepwid, onecolwid and twocolwid values, first choose how many columns you want and how much separation you want between columns
% In this template, the separation width chosen is 0.024 of the paper width and a 4-column layout
% onecolwid should therefore be (1-(# of columns+1)*sepwid)/# of columns e.g. (1-(4+1)*0.024)/4 = 0.22
% onecolwid should therefore be (1-(# of columns+1)*sepwid)/# of columns e.g. 
% (1-(3+1)*0.025)/3 = 0.3
% Set twocolwid to be (2*onecolwid)+sepwid = 0.464
% Set threecolwid to be (3*onecolwid)+2*sepwid = 0.708

\newlength{\sepwid}
\newlength{\onecolwid}
\newlength{\twocolwid}
\newlength{\threecolwid}
\setlength{\paperwidth}{36in} % A0 width: 46.8in
\setlength{\paperheight}{48in} % A0 height: 33.1in
\setlength{\textwidth}{34in} % A0 width: 46.8in
\setlength{\textheight}{46in} % A0 height: 33.1in
\setlength{\sepwid}{0.025\paperwidth} % Separation width (white space) between columns
\setlength{\onecolwid}{0.3\paperwidth} % Width of one column
\setlength{\twocolwid}{0.625\paperwidth} % Width of two columns
\setlength{\threecolwid}{0.95\paperwidth} % Width of three columns
\setlength{\topmargin}{-0.5in} % Reduce the top margin size
%-----------------------------------------------------------

\usepackage{graphicx}  % Required for including images
\newcommand{\Cyclus}{\textsc{Cyclus}\xspace}%

\usepackage{tabularx}
\newcolumntype{b}{X}
\newcolumntype{s}{>{\hsize=.5\hsize}X}
\newcolumntype{m}{>{\hsize=.75\hsize}X}
\newcolumntype{z}{>{\hsize=.65\hsize}X}

\usepackage{booktabs} % Top and bottom rules for tables
\usepackage{xspace}

\usepackage{tikz}
\usetikzlibrary{positioning, arrows, decorations, shapes, arrows.meta}
% Define block styles
\tikzstyle{decision} = [diamond, draw, fill=blue!20, 
text width=4.5em, text badly centered, node distance=3cm, inner sep=0pt]


\tikzstyle{block} = [rectangle, draw, text centered, fill=blue!20]
\tikzstyle{line} = [draw, -latex']
\tikzstyle{cloud} = [draw, ellipse,fill=red!20, node distance=6em,
minimum height=2em]



\usetikzlibrary{shapes.multipart}
\usetikzlibrary{positioning}


\setbeamertemplate{bibliography item}[text]

%----------------------------------------------------------------------------------------
%	TITLE SECTION 
%----------------------------------------------------------------------------------------

\title{
	\includegraphics[width=0.2\linewidth]{ilogo}
	\hspace{30cm}
	\vspace{2cm}
	\includegraphics[width=0.3\linewidth]{cnec_logo.png} \\
	Diversion Detection in Cyclus Archetpyes
} % Poster title

\author{\textbf{Gregory T. Westphal}, Kathryn D. Huff}
\institute{University of Illinios at Urbana-Champaign, Department of Nuclear, Plasma, and Radiological Engineering, Urbana, IL 61801}
%----------------------------------------------------------------------------------------

\begin{document}

\addtobeamertemplate{block end}{}{\vspace*{2ex}} % White space under blocks
\addtobeamertemplate{block alerted end}{}{\vspace*{2ex}} % White space under highlighted (alert) blocks

\setlength{\belowcaptionskip}{2ex} % White space under figures
\setlength\belowdisplayshortskip{2ex} % White space under equations

\begin{frame}[t] % The whole poster is enclosed in one beamer frame

\begin{columns}[t,totalwidth=\threecolwid] % The whole poster consists of three major columns, the second of which is split into two columns twice - the [t] option aligns each column's content to the top

\begin{column}{0.5\sepwid}\end{column} % Empty spacer column

\begin{column}{\onecolwid} % The first column

%----------------------------------------------------------------------------------------
%	OBJECTIVES
%----------------------------------------------------------------------------------------

\begin{alertblock}{Objectives}
\begin{itemize}
        \item Timely detection of diversion relies on the identification of signatures and observables for unique facilities. 
        \item Create high-fidelity diversion algorithms .
        \item Determine optimum detector and inspection locations in pyroprocessing facilities using the Cyclus framework.
        \item Adapt this work to be applicable to a wide range of nuclear fuel cycle facilities in cyclus
        \item Characterize required detection sensitivities and corresponding 
                false positive rates. 
\end{itemize}

\end{alertblock}

%----------------------------------------------------------------------------------------
%	BACKGROUND
%----------------------------------------------------------------------------------------

\begin{block}{Background}


\begin{figure}
	\includegraphics[width=\linewidth]{flowchart.pdf}
%	\includegraphics[width=1.0\linewidth]{fuel_cycle2.png}
        %\input{fc-diagram}
	\caption{Archetpye design of the Pyre facility \cite{Borrelli_2017}.}
\end{figure}

\end{block}

%----------------------------------------------------------------------------------------
%	DIVERSION DETECTION
%----------------------------------------------------------------------------------------

\begin{block}{Facility Simulation}

	\begin{figure}
		\includegraphics[width=0.9\linewidth]{timeseries-waste.png}
		\caption{Example material transactions every time step.}
	\end{figure}

	\begin{figure}
		\includegraphics[width=0.9\linewidth]{isotopic-comp-range.png}
		\caption{The isotopic breakdown of material transactions in the facility.}
	\end{figure}
\end{block}

%----------------------------------------------------------------------------------------

\end{column} % End of the first column

\begin{column}{\sepwid}\end{column} % Empty spacer column


%----------------------------------------------------------------------------------------

\begin{column}{\onecolwid} % The second column
%----------------------------------------------------------------------------------------
%	SIGNATURES AND OBSERVABLES
%----------------------------------------------------------------------------------------

\begin{block}{Material Detection}
        Material diversion occurs in two different modes: nefarious or operator.
        Nefarious diversion is the simplest as a bad actor will steal material at certain time steps.
        Operator diversion occurs when one of the facility operators is attempting to siphon off material
        by increasing productivity of the plant (or throughput) and taking off excess.
        
        
	\begin{block} {Nefarious Diversion}
		\begin{figure}
			\includegraphics[width=0.9\linewidth]{diversion1.png}
			\caption{Illustration of nefarious diversion of cyclus inventory \cite{Yilmaz_2016}.}
		\end{figure}
		
	\end{block}
	\begin{block} {Operator Diversion}
		\begin{figure}
			\includegraphics[width=0.9\linewidth]{op-diversion}
			\caption{Procedure for generating operator diversion values inside a simulation.}
		\end{figure}
		\begin{figure}
			\includegraphics[width=\linewidth]{refining}
			\caption{Example material balance used over a sub-process for diversion detection.}
		\end{figure}
	\end{block}
	
\end{block}


%----------------------------------------------------------------------------------------

\end{column} % End of column 2

\begin{column}{\sepwid}\end{column} % Empty spacer column

\begin{column}{\onecolwid} % The third column
	
\begin{block}{Diversion Detection}
	To maintain customization of the archetype the diversion detection algorithm will not know mean values for transactions.
	It is assumed that diversion will occur a number of time steps after startup, allowing the cumulative sum to approximate a mean
	based on simulated data.
	\begin{figure}
		\includegraphics[width=0.9\linewidth]{cusum-example.png}
		\caption{Cumulative sum method being used to detect a change in material flow \cite{Yilmaz_2016}.}
	\end{figure}
	
\end{block}
%----------------------------------------------------------------------------------------
%	FUTURE WORK
%----------------------------------------------------------------------------------------

\begin{alertblock}{Future Work}
	The goal of this poster is to outline what has been accomplished in cyclus diversion detection and review challenges and properties
	specific to pyroprocessing. What needs to be accomplished proceeding this work is as follows:
	\begin{itemize}
		\item Sensitivity analysis of diversion methods.
		\item Compare Cyclus output for various facility configurations.
		\item Assess capability of using Cyclus as online detection.
	\end{itemize} 
	\vspace{10mm}
	In addition to completing the diversion detection module for pyroprocessing, the goal is to expand this to be accessible to
	other Cyclus archetypes as well \cite{Huff_2016}. Other capabilities to be added include accounting for a variety of diversion times,
	currently the algorithm is capable of routine and set times for diversion.
\end{alertblock}

%----------------------------------------------------------------------------------------
%	ACKNOWLEDGEMENTS
%----------------------------------------------------------------------------------------

\setbeamercolor{block title}{fg=norange,bg=white} % Change the block title color

\begin{block}{Acknowledgements}
	
	This research was performed using funding received
	from the Consortium for Nonproliferation Enabling
	Capabilities under award number 1-483313-973000-191100.
	
	\vspace{10mm}
	\begin{center}
		\begin{tabular}{ccc}
			\includegraphics[width=0.3\linewidth]{logo.png} & \includegraphics[width=0.5\linewidth]{cnec_logo.png}
		\end{tabular}
	\end{center}
	
	
\end{block}

%----------------------------------------------------------------------------------------
%	CONTACT INFORMATION
%----------------------------------------------------------------------------------------

\setbeamercolor{block alerted title}{fg=black,bg=norange} % Change the alert block title colors
\setbeamercolor{block alerted body}{fg=black,bg=white} % Change the alert block body colors



\begin{alertblock}{Contact Information}
	\setbeamercolor{block title}{fg=norange,bg=white} % Change the block title color
	\begin{itemize}
		
		\item Web: \href{arfc.github.io}{arfc.github.io}
		\item Email: \href{mailto:gtw2@illinois.edu}{gtw2@illinois.edu}
		\item Phone: +1 (636) 284-9691
	\end{itemize}
	
\end{alertblock}

\begin{block}{References}

        {\footnotesize\bibliographystyle{abbrv} 
        \bibliography{poster}}
\end{block}


%----------------------------------------------------------------------------------------



\end{column} % End of the third column

\end{columns} % End of all the columns in the poster

\end{frame} % End of the enclosing frame

\end{document}
\begin{column}{\sepwid}\end{column} % Empty spacer column
